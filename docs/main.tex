% Preamble 
\documentclass[12pt]{report}
%\linespread{1.3}
\usepackage[utf8]{inputenc}
\usepackage[british,magyar]{babel} % korrekt elválasztás-
\usepackage[T1]{fontenc} % korrekt elválasztás
\usepackage{t1enc} % ez, vagy a megelőző felesleges, de ha itt van mindkettő, az nem baj

\usepackage[a4paper, top=2.5cm,right=2.5cm,left=3cm,bottom=2.5cm]{geometry} 

\usepackage{appendix}

% Basics
\usepackage{color}
\usepackage{graphicx}
\usepackage{subcaption, caption}
\usepackage{hhline}
\usepackage{wrapfig}
\usepackage{lipsum}
\usepackage{multirow}
\usepackage{{booktabs}}

\usepackage{floatrow} % for image sources

	
% Links, quotes, fancies
\usepackage{tocbibind}

\usepackage[table]{xcolor}

\usepackage{url} 		% to include any web link.
\usepackage{listings}
\newcommand*{\rom}[1]{\expandafter\@slowromancap\romannumeral #1@}
\usepackage[final]{pdfpages}
\usepackage{datetime}  % to have month-year date type for the as of date.
\newdateformat{monthyeardate}{%
	\THEYEAR.~\monthname[\THEMONTH]}

% Page style setup
\usepackage[raggedright, pagestyles]{titlesec}
\usepackage[raggedright]{titlesec}
\usepackage{fancyhdr}
\fancypagestyle{front}{%
	\fancyhf{}
	\renewcommand{\headrulewidth}{0pt}
	\fancyfoot[C]{\thepage}
}
\fancypagestyle{main}{%
	\fancyhf{}
	\fancyhead[R]{\rightmark}
	\fancyhead[L]{\leftmark}
	\fancyfoot[C]{\thepage}
	\renewcommand{\headrulewidth}{0.4pt} 
}
\fancypagestyle{plain}{%
	\fancyhf{}
	\fancyhead{}
	\fancyfoot[C]{\thepage}
	\renewcommand{\headrulewidth}{0pt}
}


\frenchspacing % magyar nyelvnek megfelelő mondatvégi térköz (pont utáni térköz csökkentése)
\linespread{1.3} 
\usepackage{indentfirst}

\usepackage{qrcode}

\usepackage{epigraph,varwidth}
\renewcommand{\epigraphsize}{\small}
\setlength{\epigraphwidth}{0.65\textwidth}
\renewcommand{\textflush}{flushepinormal}

%\usepackage{natbib}
\usepackage{import}
\usepackage[hidelinks]{hyperref}
\hypersetup{
	colorlinks,
	linkcolor={red!50!black},
	citecolor={blue!50!black},
	urlcolor={blue!80!black}
}  % reset the default color of hyperlinks

\makeatletter
\renewcommand{\sectionmark}[1]{\markright{\thesection~~~#1}}
\renewcommand{\chaptermark}[1]{\markboth{\if@mainmatter \fi#1}{}}
\makeatother

% % % renew abstract environment to avoid page renumberin
\makeatletter
\renewenvironment{abstract}{%
	\if@twocolumn
	\section*{\abstractname}%
	\else
	\small
	\begin{center}%
		{\bfseries \abstractname\vspace{-.5em}\vspace{\z@}}%
	\end{center}%
	\quotation
	\fi}
{\if@twocolumn\else\endquotation\fi}
\makeatother
\providecommand{\keywords}[1]{\small\textbf{Keywords: } #1}


% Doc Values

\title{\foreignlanguage{british}{Helpdesk} rendszer megvalósítása mikroszerviz alapú elosztott alkalmazással}
\author{Bőle Balázs}
\date{\today}
\graphicspath{{Figs/}}  % automatic graphics folder

%%%%%%%% DOCUMENT %%%%%%%%
\begin{document}
\sloppy
\pagenumbering{roman}
%%%% Title Page
%\pagestyle{empty}
\makeatletter  % referring to the values of title, etc.
\begin{titlepage}
\newcommand{\HRule}{\rule{\linewidth}{0.5mm}} 							% horizontal line and its thickness
\center 
%\flushright
% University
\textsc{\huge Gábor Dénes Főiskola}\\[1cm]

% Document info
\textsc{\LARGE Mérnökinformatikus alapképzés}\\[0.2cm]
\HRule \\[0.8cm]
{\huge \bfseries \@title}\\[0.7cm]								% Assignment
\HRule \\[2cm]

% Author
{\LARGE \bfseries
\@author}\\[1.5cm]
%Supervisor
{\Large Konzulens:\\[0.5cm]
Dr. Nagy Elemér Károly}\vspace{2cm}

{\LARGE Szoftverfejlesztés szakirány}\\[0.5cm]					    % Course Code

\includegraphics[height=0.2\textwidth]{gdf_logo.png}

\vfill
{\Large \monthyeardate{\@date}}
\end{titlepage}

\thispagestyle{empty}
\cleardoublepage

\includepdf[pages=-]{FM008_01_Szakdolgozatterv-DXQRPJ.pdf}
\cleardoublepage
\includepdf[pages=-]{eredetiseg-nyilatkozat.pdf}

\thispagestyle{empty}
\cleardoublepage

\thispagestyle{empty}
\begin{center}
	{\Large \bfseries \@title}
	
	\bigskip
	\bigskip
	
	készítette
	
	\bigskip
	\bigskip
	
	{\large \bfseries \@author}
	
	\bigskip
	
	\vfill
	
	\begin{tabular}{ll}
		Neptun kód: & DXQRPJ \\
		Elérhetőség: & \href{mailto:bolebalazs@gmail.com}{bolebalazs@gmail.com}\\
		
		\bigskip\\
		
		Konzulens: & Dr.\ Nagy\ Elemér\ Károly
		

	\end{tabular}

	\bigskip
	\bigskip

\mbox{A dolgozat elektronikus változata elérhető a \url{https://github.com/balazsBole/} címen.}	
	

	
	\bigskip
	
	\includegraphics[width=0.2\textwidth]{gdf_logo.png}

	
\end{center}
Budapest, \monthyeardate{\@date}.

\makeatother

\clearpage
\begin{abstract}
	//todo: meg kell írni az abstracot
\end{abstract}
\keywords{abstract keywords}


\pagestyle{front}
\tableofcontents

\begingroup
\let\clearpage\relax
\listoffigures
\endgroup


\clearpage
\pagenumbering{arabic}
\chapter*{Bevezetés}\label{ch:bevezetes}
	\addcontentsline{toc}{chapter}{\nameref{ch:bevezetes}}
Ahogy az \foreignlanguage{british}{O'Really} által az év elején készített felmérésből \cite{OReally} is látszik, a mikroszerviz alapú alkalmazások egyre nagyobb népszerűségnek örvendenek. Egyre több cég szeretné lecserélni meglévő monolit rendszerét, vagy a szükséges új funkciókat a régebbi rendszertől függetlenül, hibrid rendszerben valósítana meg.

Mint az a \foreignlanguage{british}{wiredelta} cikkéből \cite{wiredelta} is látszik, a mikroszerviz architektúrának számtalan előnye van. Míg a nagyvállalati környezetben sokszor a folyamatos szállítási igény, vagy az egymástól függetlenül fejleszthető alrendszerek miatt döntenek emellett a technológia mellett, az én esetemben a legfontosabb szerepet a skálázhatóság, az újrafelhasználhatóság, és  az alacsony fenntartási költség játszotta.

Úgy gondolom, hogy nincs olyan technológia, ami minden problémára megoldást nyújtana. De úgy érzem hogy az ilyen elvek mentén kialakított alkalmazások, természetükből adódóan időtállóbbak lesznek. Ha el tudjuk érni, hogy egy alkalmazás valóban csak egy funkcióért kell hogy felelős legyen, azzal a problémamegoldás analitikus oldalát  emeljük rendszerszintre. 

Éppen ezért, a mikroszerviz architektúra legnagyobb előnye szerintem a rendszerezésből következik.


\chapter{Üzleti igények}\label{ch:uzleti_igenyek}
\pagestyle{main}
Ebben a fejezetben szeretném bemutatni a \foreignlanguage{british}{Helpdesk} alkalmazás felé megfogalmazott üzleti igényeket.



\section{Funkcionális igények}

\subsection{E-mail fogadása és küldése}\label{sec:email_fogadas_kuldes}
Az ügyfelektől érkező e-maileket az alkalmazás képes fogadni, hosszú távra megőrizni. Számukra formázott válasz e-mail küldhető.

A rendszernek képesnek kell lennie több e-mail cím kezelésére. A beérkező új üzeneteket a címzettnek megfelelő előre definiált sorhoz kell hozzárendelni.



\subsection{E-mail szálak kezelése}
A rendszer által kezelt üzenetek szálakba rendezve érhetőek el. Egy szál az ügyfél és a felhasználó közötti üzenetváltásokból épül fel.

Az üzenetszálakra vonatkozó összes adat historikusan lekérdezhető, státuszuk \aref{fig:statusz_diagram} ábrán definiált útvonalaknak megfelelően változtatható.  

\begin{figure}[hbt] 
	\caption{Az e-mailszálak státuszváltozásai}
	\centering
	\includegraphics[width=0.75\textwidth]{statusz_diagram_drawio.pdf}\label{fig:statusz_diagram}
\end{figure}



\subsection{Több felhasználó}
A rendszert egyszerre több felhasználó használhatja. Minden felhasználó csak a saját emailszálait kezelheti, csak azokra válaszolhat.

	
Minden felhasználó pontosan egy \aref{sec:email_fogadas_kuldes} fejezetben említett sorhoz tartozik. Csak az ugyanabba a sorba tartozó e-mail szál rendelhető hozzá.
A számára kijelölt szálakat képes --a saját során belül-- más felhasználóhoz rendelni. 

A felhasználók eltérő jogkörökkel rendelkezhetnek. Az adminisztratív jogkörrel rendelkező felhasználó végzi az új emailszál felhasználóhoz rendelését, valamint a \foreignlanguage{british}{\textit{change queue}} státuszban (\ref{fig:statusz_diagram} ábra) lévő üzenetszálak új sorba irányítását.

A konfigurációs jogkörrel rendelkező felhasználó feladata más felhasználók regisztrálása, valamint az alakalmazásban használt jogkörök (\foreignlanguage{british}{\textit{role}}-ok) kezelése. Lehetősége van továbbá authentikációs logok megtekintésére, jelszó visszaállítására és más felhasználók megszemélyesítésére (\foreignlanguage{british}{\textit{impersonate}}).

A felhasználói felületen elérhető funkciókat \aref{fig:funkcio_diagram} ábra foglalja össze.

\begin{figure}[hbt] 
	\caption{Elérhető funkciók jogosultság szerint csoportosítva}
	\centering
	\includegraphics[width=0.75\textwidth]{funkcio_diagram_drawio.pdf}\label{fig:funkcio_diagram}
\end{figure}



\subsection{I18N}
A felhasználói, adminisztratív és karbantartói felületek angol nyelven érhetőek el. Több nyelv kezelése nem szükséges.


\section{Nem funkcionális igények}	
\lipsum


angol nyelven elérhető


\chapter[Technológiai áttekintés]{Felhasznált technológiák}\label{ch:felhasznalt_technologiak}
\pagestyle{main}
Az alkalmazás rendszer szinten mikroszerviz (\ref{sec:mikroszerviz}), a modulok szintjén hexagonális architektúrába (\ref{sec:hexagonalis_architektura}) rendezve készült el. A frontend Angulart (\ref{sec:angular}), a backend és az e-mail kliens Spring Boot-ot (\ref{sec:spring_boot}) használ. A alkalmazáson belüli események kezelésére és tárolására Apache Kafkát (\ref{sec:apache_kafka}) használok.


\section{Mikroszerviz architektúra}\label{sec:mikroszerviz}
Bár a kifejezés már régóta ismert, nincs egy központilag elfogadott, egységes definíció arra nézve, miket nevezünk mikroszervizeknek. A legtöbb szerző jobb híján a visszatérő karakterisztikus tulajdonságuk alapján sorolja be az alkalmazásokat ebbe a kategóriába~\cite{OReally_Microservice_Architecture}. Egy tipikus mikroszerviz a következő tulajdonságoknak felel meg:

\begin{itemize}
	\item	pontosan egy üzleti funkció köré szerveződik,
	\item   más	szolgáltatásokkal laza, általában hálózaton keresztül megvalósuló kapcsolatban áll,
	\item   ha szüksége van adatbázisra, akkor sajáttal rendelkezik, más rendszer ezt az adatbázist nem éri el,
	\item	önmagában is működőképes,
	\item	decentralizált, tehát nincs egy a munkáját befolyásoló központi irányítórendszer.
\end{itemize}

A hasonló felépítésükből adódóan, számos olyan eszköz van, ami --~nem kötelezően, de legtöbbször~--   együtt fordul elő a mikroszerviz architektúrával. A legfontosabb ilyen fogalmak a:
\begin{description}
	\item[skálázhatóság] a rendszer képessége az áteresztőképességének növelésére.
	Létezik vertikális\footnote{több processzor vagy memória bevonása} és horizontális skálázhatóság\footnote{újabb példányok futtatása}.
	
	\item[konténerizálás] az adott szolgáltatás futtatása saját elszeparált környezetében hardveres virtualizáció segítsége nélkül.	

	\item[szolgáltatás felderítés] a rendszer által nyújtott szolgáltatások, szervizek automatikus
	felfedezhetősége\footnote{angolul \emph{service discovery}-nek hívják}.
	
	\item[loadbalancer] az a folyamat, ami a bejövő feladatokat erőforrásokhoz rendeli. Legegyszerűbb megvalósítása a \foreignlanguage{british}{\emph{round robin}} algoritmus, célja a terhelés egyforma elosztása.
	
	\item[monitorozás] az önálló szolgáltatások állapotának felügyelése. A monitorozás során nyújtott metrikák kiterjedhetnek a felhasznált memória mennyiségére, processzorigényére, vagy processzeire is.
\end{description}


\section{Hexagonális architektúra}\label{sec:hexagonalis_architektura}
A hexagonális architektúra --~vagy más néven portok és adapterek architektúrája~--   egy \foreignlanguage{british}{Alistair Cockburn} által létrehozott \cite{Alistair_Cockburn} szoftverterezési minta. Nevét a cikkben felrajzolt hatszögletű rendszerábrázolásról kapta (\ref{fig:Alistair_Cockburn_hexagonal_architecture} ábra), ami szembemegy a korábban elterjedt réteges elrendezéssel.

Az eredeti szándék mögötte az alkalmazás függetlenítése mindennemű külső függőségtől\footnote{például adatbázis, felhasználók, automatizált tesztek}, így lehetővé téve az üzleti és a technikai igények nagy mértékű szeparálását.
Egy absztrakt port feladata kell legyen a külvilággal való kapcsolat, így az üzleti logika csak az üzenet tartalmáért felelős, az üzenetküldés módjáért már nem.

\begin{figure}[hbt] 
	\centering
	\includegraphics[width=0.85\textwidth]{Alistair_Cockburn_hexagonal_architecture.png}
	\caption[Hexagonális alkalmazások felépítése]{
		A hexagonális alkalmazás külső függőségeinek elszeparálása}
	\label{fig:Alistair_Cockburn_hexagonal_architecture}
	\floatfoot{Forrás: Alistair Cockburn~\cite{Alistair_Cockburn}}
\end{figure}



Ahogy \foreignlanguage{british}{Robert C. Martin} a \foreignlanguage{british}{\emph{The Clean Architecture}} című cikkében \cite{The_Clean_Architecture} összeszedte, a port-adapter és a hasonló architektúrával készülő alkalmazások mind:	
\begin{itemize}
	\item Könnyen, és önmagukban is tesztelhetőek, mivel az üzleti szabályoknak nincs külső függőségük.
	
	\item Függetlenek a külső tényezőktől. Így az alkalmazás által használt felület vagy adatbázis könnyen cserélhető.
	
	\item Keretrendszertől függetlenül is megvalósíthatóak. A megvalósítás nem függ semmilyen könyvtártól vagy egyéb tulajdonságtól.
\end{itemize}
	


\section{Rétegek szeparálása}\label{sec:retegek_szeparalasa}
A hexagonális architektúra (\ref{sec:hexagonalis_architektura} pont) és a hasonló \emph{clean code} \cite{clean_code_chapter_systems} elvek sokszor a különböző szoftver rétegek elkülönítésén alapszanak.


Annak érdekében, hogy a feladatok elkülönítése ne vonzza magával az ismétlődő program részletek megnövekedését, célszerű generálni a visszatérő, üzleti funkciót nem hordozó sorokat. Ilyen --~a fordítási időben~--   kódot generáló eszköz a Mapstruct és a Lombok.


\section{Konkurencia kezelése}\label{sec:konkurencia_kezekese}
Ha az alkalmazásnak egyszerre több felhasználót kell kiszolgálnia, vagy bármilyen oknál fogva ugyanazt az adatot egy időben több program módosítaná, akkor az inkonzisztens állapot elkerülése érdekében célszerű valamilyen konkurenciakezelési stratégiát alkalmazni. Alapvetően két fajta konkurenciakezelő megoldás létezik:

\begin{description}
	\item[optimista] konkurenciakezelést akkor érdemes használni, mikor számíthatunk arra, hogy az esetek többségében nincs párhuzamos módosítás. Ütközés esetén --~ha egyszerre kellene ugyanazt az adatot módosítani~--   a tranzakciót elvetjük és értesítjük a módosítást kezdeményező felet, hogy időközben  az adat megváltozott.
	
	Ez a megoldás tehát nagy mennyiségű adat, és hozzá képest relatív kis számú felhasználó esetén ideális.
	
	\item[pesszimista] konkurenciakezelés esetén, a módosítani kívánt adatot olvasáskor zároljuk, az csak a módosítás befejezése után lesz újra hozzáférhető a többi fél számára.
	
	Az adatok a teljes tranzakció ideje alatt zárolva vannak, ezért ez a megoldás gyakran jár együtt teljesítménycsökkenéssel. A kölcsönös zárolás pedig, --~mikor két vagy több tranzakció egymás befejezésére vár~--   könnyen vezethet \emph{deadlock}hoz.
\end{description}


\section{Alkalmazások szeparálása}\label{sec:alkalmazasok_szeparalasa}
Ahogy azt \aref{sec:mikroszerviz}. pontban is írtam, hogy megvalósítható legyen a szolgáltatások laza kapcsolata és egymástól független működése, a mikroszerviz csak a saját adatbázisához férhet hozzá. Ez lehetővé teszi a feladatnak megfelelő adatbázis választását is.

A mikroszervizeken átnyúló üzleti funkciók megvalósítására több megoldás is létezik:
\begin{description}
	\item[API kompozíció] A legegyszerűbben megvalósítható az API kompozíció. Ebben az esetben az applikáció maga végzi el, saját memóriájában az adatok egymáshoz rendelését. 
	
	Kis számú adatnál használható, és célszerű elkerülni hogy az adat kettő vagy annál több számú mikroszervizen keresztül érkezzen meg.
	
	\item[CQRS] A CQRS\footnote{Command Query Responsibility Segregation} az  olvasás és írás műveletének elszeparálásán alapuló megoldás~\cite{OReally_Microservice_Architecture_CQRS}. Lényege hogy a CRUD műveletekről minden esetben egy esemény keletkezik. Ezekre az eseményekre bármelyik mikroszerviz feliratkozhat.
	
	Ha más rendszernek szüksége van az aktuális állapotra, az az események újrajátszásával bármikor megkapható.	
	 
	Az Apache Kafkát (\ref{sec:apache_kafka}) gyakran használják az események kezelésére, mert natívan támogatja az események csoportosítását egyedi azonosító alapján. Beállítható, hogy UUID alapján mindig csak a legfrissebb állapot legyen elérhető, ezzel lecsökkentve a kezdeti olvasáshoz szükséges időt.
		
	\item[Elosztott tranzakciók és Saga] Ha nem csak más szolgáltatások adatainak olvasásáról van szó, hanem több szolgáltatáson átívelő, visszagörgethető tranzakciókat kell megvalósítani, arra az esetre találták ki a \emph{Saga}-t.
	
	A \emph{Saga} egy hosszú életű elosztott tranzakció~\cite{OReally_Microservice_Architecture_Saga}. A folyamat lépései sorban hajtódnak végre, minden lépés tartalmaz egy utasítást arra az esetre ha vissza kellene görgetni a teljes folyamatot. Ha a folyamat bármelyik lépésnél meghiúsul, onnantól fogva visszafelé minden rendszer egyesével visszaáll a tranzakció előtti állapotra.
\end{description}


\section{Apache Kafka}\label{sec:apache_kafka}
Az Apache Kafka egy üzenet tárolásra és továbbításra kifejlesztett hibatűrő, magas áteresztő képességű, nyílt forráskódú alkalmazás~\cite{OReally_Kafka}.

A feladó az üzenetet nem közvetlenül a fogadónak küldi, hanem egy üzenetbrókeren keresztül egy (\emph{topic})-ba teszi közzé. A fogadó fél dönti el, hogy melyik téma üzeneteit szeretné megkapni. Annak érdekében, hogy megkaphassa az üzenetet, feliratkozik az üzenet témájára.

Redundancia és skálázhatóság miatt egy \emph{topic} több partícióra van elosztva, és ezen felül minden partíció replikálva is van~\cite{OReally_Kafka_Internals}. A partíciók eltérő szerveren lehetnek, ezáltal egy \emph{topic} horizontálisan skálázható. Egy szerver esetleges kiesése esetén a többi szerver át tudja venni a kiesett szerver szerepét.

Az üzenetbrókerek összehangolását a Zookeper szolgáltatás végzi. Mivel minden kafka bróker beregisztrálja magát a szolgáltatásba, így a Zookeper mindig naprakész információval rendelkezik az üzenetbrókerekről.

Az üzeneteket Apache Avroval szerializálom. Az Avro lehetővé teszi a kompakt bináris tárolást, de natívan támogatja a JSON reprezentációt is. Az Avrohoz szükséges séma nyilvántartásért és az eltérő verziók kezelésért a Schemaregistry szerver felelős. A kafka kliensek a Schemaregistry szerveren keresztül tudják az üzeneteket olvasni és írni. 



\section{Angular}\label{sec:angular}
Az \foreignlanguage{british}{Angular} egy a \foreignlanguage{british}{Google} által fejlesztett \foreignlanguage{british}{TypeScript} alapú platform és	keretrendszer~\cite{angular_docs}. A segítségével létrehozott kód erősen modularizált, így könnyű vele újra felhasználható és a modell-nézet-vezérlő elvet követő alkalmazást létrehozni.

Az Angularral készített honlap teljes mértében a kliens oldalon fut, így a szerver oldalon elegendő egy egyszerű, statikus HTML-oldalt visszaadó alkalmazásszerver használata.


\section{Spring Boot}\label{sec:spring_boot}
A \foreignlanguage{british}{Spring Boot}, egy a Springre épülő keretrendszer. Mindkét rendszer alapja a függőség befecskendezése\footnote{Angolul \foreignlanguage{british}{\emph{Dependency Injection}}}, ami egy \aref{sec:retegek_szeparalasa} pontban említett tiszta kód \cite{clean_code_chapter_systems} eszköze.

A Spring Boot \cite{introducing_spring_boot} célja, hogy gyorsan és egyszerűen lehessen önálló, magas minőségű alkalmazásokat fejleszteni:
\begin{itemize}
	\item az alapbeállítástól való eltérést kell meghatározni\footnote{A Spring Boot dokumentációban ezt röviden \emph{convention over configuration}-nek hívják} ezzel lecsökkentve a konfigurációval töltött időt,
	\item valamint sok gyakran visszatérő problémára\footnote{Például: metrikák, biztonság, adattárolás} nyújt könnyen elérhető megoldást.
\end{itemize}






\chapter{Az alkalmazás felépítése}\label{ch:felepites}
\pagestyle{main}
Ebben a fejezetben átfogó képet adok az általam létrehozott helpdesk alkalmazásról. Az egyes komponensek részletes leírása \aref{ch:implementacio}. fejezetben található.

\section{Legfontosabb komponensek}

\begin{figure}[hbt] 
	\centering
	\includegraphics[width=0.87\textwidth]{komponens_diagram_drawio.pdf}
	\caption{A legfontosabb komponensek}
	\label{fig:komponens_diagram}
	\floatfoot{Forrás: saját ábra}
\end{figure}

\Aref{fig:komponens_diagram}. ábrán a legfontosabb szolgáltatásokat gyűjtöttem össze. Az üzleti funkcionalitás megvalósulása az itt bemutatott komponensek összehangolt munkáján keresztül valósul meg.



\begin{itemize}
	\item A felhasználó az nginx-en (\ref{sec:nginx} pont) keresztül éri el a heldesk alkalmazást.
	\item Az nginx dönti el, hogy melyik URL-t melyik szolgáltatás szolgálja ki.
	\item Az email kliens és a heldpesk backend kafka streamen keresztül éri el egymást.
	\item Az email kliensek kezelik az e-mail szerverekkel való adatcserét.
\end{itemize}


\section{Adatbázis UML diagram}
A helpdesk backend adatbázis legfontosabb tábláit \aref{fig:basic_database_uml}. ábra tartalmazza. Az ábrán nem szerepelnek az audit és a Liquibase által használt táblák~(\ref{sec:adatbazis} pont). \Aref{ch:bemutatas}. fejezetben található \ref{fig:extended_database_uml}. ábra tartalmazza az adatbázis összes tábláját.
 


\begin{figure}[hbt] 
	\centering
	\includegraphics[width=0.85\textwidth]{basic_database_uml.png}
	\caption{A backend legfontosabb adatbázistáblái}
	\label{fig:basic_database_uml}
	\floatfoot{Forrás: saját ábra}
\end{figure}

\section{E-mail fogadásának és küldésének folyamata}
A könnyebb átláthatóság érdekében, a folyamatokat egy e-mail szemszögéből mutatom be \aref{fig:path_of_an_email}. ábrán. Az e-mail fogadása az alábbi felsorolás által leírt folyamat szerint történik.
\begin{enumerate}
	\item Az e-mail kliens IMAP protokollon keresztül megkapja az új e-mailt.
	\item Az e-mail kliens a bejövő e-mailt egy kafka üzenetként teszi közzé a bejövő e-mailek kafka \textit{topicban}.
	\item A bejövő e-mailek \textit{topic}ra feliratkozott helpdesk backend megkapja a kafka üzenetet.
	\item A helpdesk backend eltárolja az új üzenetet az adatbázisban
	\item A felhasználó a frontend segítségével lekérdezi az újonnan beérkezett e-maileket.
	\item A helpdesk backend a kérésre elküldi az újonnan fogadott e-mailt.
\end{enumerate}

\bigskip
Az email küldése pedig az alábbi felsorolás szerint hajtódik végre.
\begin{enumerate}
	\item A felhasználó az új e-mail elolvasása után a frontend segítségével megírja a választ.
	\item A felhasználó elküldi a választ a helpdesk backendnek.
	\item A helpdesk backend eltárolja az adatbázisba az új e-mailt, majd az e-mail szálnak megfelelő kimenő e-mail \textit{topic}ba közzéteszi az új üzenetet.
	\item Az e-mail cím specifikus kimenő e-mailek \textit{topic}ra feliratkozott e-mail kliens megkapja a kafka üzenetet.
	\item Az e-mail kliens SMTP protokollon keresztül elküldi az új e-mailt.
\end{enumerate}


\begin{figure}[hbt] 
	\centering
	\includegraphics[width=0.7\textwidth]{path_of_an_email_drawio.pdf}
	\caption{A bejövő és kimenő e-mail útja}
	\label{fig:path_of_an_email}
	\floatfoot{Forrás: saját ábra}
\end{figure}



\chapter{Implementáció}\label{ch:implementacio}
\pagestyle{main}
\section{Mikroszerviz infrastruktúra}
\subsection{nginx}
static html-t szolgál ki a frontendnek -- lightweit application server

routingot valósul meg, rajta keresztül lehet elérni az auth*-ot és a helpdesk backendet

cache-t valósít meg a backen és a frontend között

\subsection{docker konténerizáció}
round-robin dns, docker compose, scale, images

\subsection{metrikák}
prometheus scraper, graphana, eurekát is itt használom leginkább






\section{E-mail kliens}
kafka producer, imap és smtp kliens

\subsection{E-mail szabvány}
	rfc5322: 
	messageId
	replyTo
	refereneces







\section{helpdesk backend}
Class Diagram

\subsection{springBoot}
default behavior, DI
security, 	

\subsection{data persistance layer}
hibernate, liquibase, Hibernate envers (Auditlog)

\subsection{egyéb eszközök}
openApi dokumnetáció, mapstruct, hibernate, lombok






\section{helpdesk frontend}
MVC szerint van szeparálva a kód 	

\subsection{RxJs store}

\subsection{Komponensek}
material, quill

\subsection{deploymnet}
static html-lé fordul a kliens oldalán fut a frontend






\section{keycloak}	
\subsection{JWT-token}
\subsection{role-ok}
\subsection{admin felületről valami?}


\chapter{Alkalmazás bemutatása}\label{ch:bemutatas}
\pagestyle{main}
A helpdesk alkalmazás \aref{ch:uzleti_igenyek}. fejezetben leírtaknak megfelelően szolgál ki három különböző e-mail címet:

\begin{itemize}
	\item a \textit{generic} sorhoz tarozó  \href{mailto:helpdesk.gdf@yandex.com}{\nolinkurl{helpdesk.gdf@yandex.com}}-ot, 
	\item a \textit{travel} sorhoz tarozó  \href{mailto:helpdesk.gdf.travel@yandex.com}{\nolinkurl{helpdesk.gdf.travel@yandex.com}}-ot,
	\item és a \textit{theater} sorhoz tarozó  \href{mailto:h.gdf.theater@gmx.com}{\nolinkurl{h.gdf.theater@gmx.com}}-ot.
\end{itemize}



\section{Alkalmazás elindítása}\label{sec:elinditas}
Az alakalmazás a \textit{start.sh} bash \textit{script}tel indítható el. A \textit{script} két dolgot csinál:
\begin{enumerate}
	\item a \textit{docker-compose} parancssal elindítja a docker \textit{container}eket~(\ref{sec:docker} pont),
	\item  ``helpdesk'' domain névvel hozzáadja az Nginx (\ref{sec:nginx} pont) IP címét a \mbox{\textit{/etc/hosts}} állományhoz.
\end{enumerate}

A \textit{script} indítása után a helpdesk alkalmazás elérhető a  \href{http://helpdesk}{http://helpdesk} domain alatt.


\section{Több példány}
A különböző szervizekből a terhelésnek megfelelően eltérő számú példány indul el:

\begin{itemize}
	\item a helpdesk-backendből három,
	\item a \textit{theater} sort kezelő email-kliensből egy,
	\item a \textit{travel} sort kezelő email-kliensből kettő,
	\item a \textit{generic} sort kezelő email-kliensből három,
	\item és a Kafka brókerből (\ref{sec:kafka_topics}) szintén három darab.
\end{itemize}

A példányok metrikáit (\ref{sec:metrikak} pont) nyomon lehet követni az erre a célra létrehozott Grafana oldalon (todo ábra). Az oldal elérhető a  \textit{Spring metrics} menüpont alatt.

Az todo ábrán csak a Grafana oldal legfelső néhány panele látható, az instance-okra lebontott legfontosabb mérőszámokkal:

\begin{itemize}
	\item a legfelső sorban a Java Virtual Machine, által akutálisan felhasznált Heap space,	
	\item alatta az aktuális REST lekérések száma,
	\item a harmadik sorban a feldolgozott Kafka üzenetek száma,
	\item míg az utolsó sorban a Trace log bejegyzések száma látható.
\end{itemize}


\section{Deployment}
A könyebb bemutathatóság érdekében a szemléletesebb szervizeket --hogy ne a docker daemon által kiosztott IP címen keresztül kelljen elérni-- a docker hálózaton kívül is elérhetővé tettem. 

A \textit{docker-compose} (\ref{sec:elinditas}) által elindított \textit{container}eket \aref{fig:deployment_diagram}. ábrán foglaltam össze. Az ábrán feltüntettem, hogy az adott \textit{container}t a \textit{localhost} melyik portján lehet elérni.


\begin{figure}[hbt] 
	\centering
	\includegraphics[width=0.85\textwidth]{deployment_diagram_drawio.pdf}
	\caption[Deployment diagram]{Deployment diagram}
	\label{fig:deployment_diagram}
	\floatfoot{Forrás: saját ábra}
\end{figure}


\section{E-mail fogadásának és küldésének folyamata}
\Aref{ch:felepites}. fejezetben \aref{fig:path_of_an_email}. ábrán bemutattam egy e-mail fogadásának elméleti útját. Most \aref{fig:email_send_receive_visible}. ábrán szeretném bemutatni hogyan követhető végig a rendszerben egy e-mail valódi útja.

E-mail fogadása:
\begin{enumerate}
	\item Egy teszt üzenet érkezik a \href{mailto:helpdesk.gdf.travel@yandex.com}{\nolinkurl{helpdesk.gdf.travel@yandex.com}} címre
	\textit{E-mail fogadásának és küldésének folyamata} tárggyal
	\item Az e-mail kliens kafka üzenetként publikálja az üzenetet az \textit{email.in.v1.pub} topicba (\ref{fig:email_client_send_kafka}. ábra).
	\item A backend megkapja a kafka üzenetet (\ref{fig:backend_receive_kafka}. ábra)
	\item A backend elmenti az új üzenet az adatbázisba (\ref{fig:datbase_received_email}. ábra)
	\item A felhasználói felületen (\ref{fig:frontend_read_email}. ábra) elérhető az új üzenet.
\end{enumerate}

\bigskip

E-mail küldése:
\begin{enumerate}
	\item A felhasználó elküldi a válaszát a felhasználói felületen (\ref{fig:frontend_send_answer}. ábra).
	\item A backend megkapja az üzenetet és eltárolja az adatbázisba (\ref{fig:database_answer}. ábra).
	\item A \textit{h.gdf.theater\textunderscore gmx.com.v1.pub} topicban megjelenik  (\ref{fig:kafka_topic_send_email}. ábra) a backend által publikált kafka üzenet
	\item A topicra feliratkozott e-mail kliens fogadja és továbbítja az üzenetet (\ref{fig:email_client_receives_kafka}. ábra).
\end{enumerate}
 

\begin{figure}
	\begin{subfigure}{.49\textwidth}
		\centering
		\includegraphics[width=.9\linewidth]{email_client_sending_kafka_message.png}  
		\caption{Az egyes instance kafka üzenet küld}
		\label{fig:email_client_send_kafka}
	\end{subfigure}
	\begin{subfigure}{.49\textwidth}
		\centering
		\includegraphics[width=.9\linewidth]{helpdesk_frontend_send_email.png}  
		\caption{A felületen válasz e-mailt küld a felhasználó}
		\label{fig:frontend_send_answer}
	\end{subfigure}
	
	\quad
	
	\begin{subfigure}{.49\textwidth}
		\centering
		\includegraphics[width=.9\linewidth]{backend_receiving_email_from_kafka.png}  
		\caption{Az egyes instance kafka üzenet fogad}
		\label{fig:backend_receive_kafka}
	\end{subfigure}
	\begin{subfigure}{.49\textwidth}
		\centering
		\includegraphics[width=.9\linewidth]{database_with_the_answer.png}  
		\caption{A válasz e-mail az adatbázisban}
		\label{fig:database_answer}
	\end{subfigure}

	\quad

\begin{subfigure}{.49\textwidth}
	\centering
	\includegraphics[width=.9\linewidth]{databse_incoming_email.png}  
	\caption{Az új e-mail az adatbázisban}
	\label{fig:datbase_received_email}
\end{subfigure}
\begin{subfigure}{.49\textwidth}
	\centering
	\includegraphics[width=.9\linewidth]{kafka_topic_with_the_sent_email.png}  
	\caption{A \textit{h.gdf.theater\textunderscore gmx.com.v1.pub} topic új üzenete}
	\label{fig:kafka_topic_send_email}
\end{subfigure}

	\quad

\begin{subfigure}{.45\textwidth}
	\centering
	\includegraphics[width=.9\linewidth]{helpdesk_frontend_receive_incoming_email.png}  
	\caption{A felületen elérhető az új e-mail}
	\label{fig:frontend_read_email}
\end{subfigure}
\begin{subfigure}{.45\textwidth}
	\centering
	\includegraphics[width=.9\linewidth]{email_client_sending_email.png}  
	\caption{Az egyes instance kafka üzenet fogad}
	\label{fig:email_client_receives_kafka}
\end{subfigure}

	\caption{E-mail fogadásának és küldésének folyamata során követhető lépések}
	\label{fig:email_send_receive_visible}
\end{figure}



\section{Adatbázis táblák}




\section{Apache Kafka}\label{sec:kafka_topics}
A kafka \textit{topic}ok és üzenetek elérhetőek és követhetőek a \textit{Kafka messages} menü pontja alatti Kafka Topics UI (\ref{fig:Kafka_Topics_UI}. ábra) eszközzel.

\begin{figure}[hbt] 
	\centering
	\includegraphics[width=0.9\textwidth]{Kafka_topic_ui.png}
	\caption{A Kafka Topics UI eszközzel követhetőek a kafka \textit{topic}ok üzenetei, partíciói és beállításai}
	\label{fig:Kafka_Topics_UI}
	\floatfoot{Forrás: saját ábra}
\end{figure}


A helpdesk alakalmazás összesen öt \textit{topic}-ot használ:
\begin{description}
	\item[email.in.v1.pub] Az összes beérkező  e-mailt tartalmazza.
	
	\item[helpdesk.gdf\textunderscore yandex.com.v1.pub] A  \href{mailto:helpdesk.gdf@yandex.com}{\nolinkurl{helpdesk.gdf@yandex.com}} címre küldött e-maileket tartalmazza.
	
	\item[helpdesk.gdf.travel\textunderscore yandex.com.v1.pub] A  \href{mailto:helpdesk.gdf.travel@yandex.com}{\nolinkurl{helpdesk.gdf.travel@yandex.com}} címre küldött e-maileket tartalmazza.
	
	
	\item[h.gdf.theater\textunderscore gmx.com.v1.pub] A \href{mailto:h.gdf.theater@gmx.com}{\nolinkurl{h.gdf.theater@gmx.com}} címre küldött e-maileket tartalmazza.
	
	\item[\textunderscore schemas] A Schemaregistry ebben a \textit{topic}ban tárolja az alkalmazásban használt Avro schemákat.
\end{description}


Az alkalmazás három kafka brókert futat egy clusterben. Továbbá minden üzleti funkcionalitást hordozó \textit{topic} --a \textunderscore schemas-on kívül mindegyik-- három particíóval és kettes replikációs faktorral lett létrehozva.
Így a kafka cluster egy bróker kiesése, vagy egy partíció sérülése esetén is működőképes marad.


\section{frontend}
egy-egy érdekesebb problémásabb felület
 képenyőképek? valami how to dokumentáció	
 gondolok itt az urlben állapottárolásra
 meg a user keresés leütöm vallback meg 


\section{eureka meg grafana}




\chapter{Felhasználói kézikönyv}
\pagestyle{main}
% képenyőképek? valami how to dokumentáció	
% képenyőképek? valami how to dokumentáció	


\chapter{Továbbfejlesztési lehetőségek}
\pagestyle{main}
\section{A deploymentről}

A helpdesk alkalmazás szervizei úgy lettek kialakítva, hogy képesek legyenek egymástól függetlenül, akár több példányban is működni. Ezáltal költséghatékonnyá téve a működést, leegyszerűsítve a hibatűrést és lehetővé téve a változó terhelés miatti skálázhatóságot.

Ám amíg a docker konténerek egy host gépen futnak, és egy erőforráson osztoznak, soha nem lehet gazdaságosan megoldani a skálázást, és nem tud a rendszer felkészülni a számítógép kiesésére.

A következő logikus lépés tehát az alkalmazás \textit{cluster}re migrálása. A docker natívan támogatja a Microsoft Azure és az Amazon \cite{docker_website_deploy_ECS} szolgáltatókat. Így tehát a kód és a beállítások módosítása nélkül lehetséges az alkalmazás \textit{cluster}esítése docker swarmmal.


\section{A kódról}
A helpdesk alkalmazásba --architektúrája miatt-- könnyű új funkciót fejleszteni. A most működő modulok mind lazán kapcsolódnak egymáshoz, így könnyű egy teljesen különböző, akár eltérő programnyelven íródott új funkció integrálása.

Mivel az összes technikai megkötés csupán a protokollok megvalósítása, nyugodtan lehet az új funkció tervezésénél a feladathoz választani a programnyelvet vagy a programozási módszertant is. 

Ugyanígy, a laza kapcsolatok, és jól definiált határok miatt, egyszerű egy-egy modult teljesen lecserélni, vagy más nyelven, más technológiával újraírni.

Mivel egy szerviz egy feladattal foglalkozik, ha például le kell cserélni a fontendet, akkor az új felhasználó felületen csak a megjelenítéssel kell foglalkozni, az üzleti funkciók megvalósítása a backend feladta, így azok továbbra is változatlanok maradnak.

Ugyanez nem csak a szervizek, hanem a kód szintjén is igaz. A hexagonális architektúra miatt, az adatbázis --mint külső függőség-- könnyen cserélhető. 







\chapter{Tapasztalatok}
\pagestyle{main}
Szeretném összefoglalni a szakdolgozat készítése során nyert tapasztalataim.



meg miket tanultam:
normalizálás adatbázis
spring  angular
dependency injection
multimodul maven
kafka működése
nem funkcionális technológiák, mint git, latex,  diagrammok 

kutatómunka a tervezés hogy mwnnyi minden

a problémát analizálni és megfeelő megoldást keresni rá

a végé pedig hogy ez mennyire csodás érzés hogy minden úgy működik ahogy kitaláltam.





\chapter*{Összefoglalás}\label{ch:osszefoglalas}
\pagestyle{plain}
összefoglalás a végén

meg miket tanultam:
normalizálás adatbázis
spring  angular
dependency injection

\newpage


% References
\bibliographystyle{unsrturl}
\bibliography{references.bib}



\end{document}
