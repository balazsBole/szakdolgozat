Ebben a fejezetben szeretném bemutatni a \foreignlanguage{british}{Helpdesk} alkalmazás felé megfogalmazott üzleti igényeket.



\section{Funkcionális igények}

\subsection{E-mail fogadása és küldése}\label{sec:email_fogadas_kuldes}
Az ügyfelektől érkező e-maileket az alkalmazás képes fogadni, hosszú távra megőrizni. Számukra formázott válasz e-mail küldhető.

A rendszernek képesnek kell lennie több e-mail cím kezelésére. A beérkező új üzeneteket a címzettnek megfelelő előre definiált sorhoz kell hozzárendelni.



\subsection{E-mail szálak kezelése}
A rendszer által kezelt üzenetek szálakba rendezve érhetőek el. Egy szál az ügyfél és a felhasználó közötti üzenetváltásokból épül fel.

Az üzenetszálakra vonatkozó összes adat historikusan lekérdezhető, státuszuk \aref{fig:statusz_diagram} ábrán definiált útvonalaknak megfelelően változtatható.  

\begin{figure}[hbt] 
	\caption{Az e-mailszálak státuszváltozásai}
	\centering
	\includegraphics[width=0.75\textwidth]{statusz_diagram_drawio.pdf}\label{fig:statusz_diagram}
\end{figure}



\subsection{Több felhasználó}
A rendszert egyszerre több felhasználó használhatja. Minden felhasználó csak a saját emailszálait kezelheti, csak azokra válaszolhat.

	
Minden felhasználó pontosan egy \aref{sec:email_fogadas_kuldes} fejezetben említett sorhoz tartozik. Csak az ugyanabba a sorba tartozó e-mail szál rendelhető hozzá.
A számára kijelölt szálakat képes --a saját során belül-- más felhasználóhoz rendelni. 

A felhasználók eltérő jogkörökkel rendelkezhetnek. Az adminisztratív jogkörrel rendelkező felhasználó végzi az új emailszál felhasználóhoz rendelését, valamint a \foreignlanguage{british}{\textit{change queue}} státuszban (\ref{fig:statusz_diagram} ábra) lévő üzenetszálak új sorba irányítását.

A konfigurációs jogkörrel rendelkező felhasználó feladata más felhasználók regisztrálása, valamint az alakalmazásban használt jogkörök (\foreignlanguage{british}{\textit{role}}-ok) kezelése. Lehetősége van továbbá authentikációs logok megtekintésére, jelszó visszaállítására és más felhasználók megszemélyesítésére (\foreignlanguage{british}{\textit{impersonate}}).

A felhasználói felületen elérhető funkciókat \aref{fig:funkcio_diagram} ábra foglalja össze.

\begin{figure}[hbt] 
	\caption{Elérhető funkciók jogosultság szerint csoportosítva}
	\centering
	\includegraphics[width=0.75\textwidth]{funkcio_diagram_drawio.pdf}\label{fig:funkcio_diagram}
\end{figure}



\subsection{I18N}
A felhasználói, adminisztratív és karbantartói felületek angol nyelven érhetőek el. Több nyelv kezelése nem szükséges.


\section{Nem funkcionális igények}	
\lipsum


angol nyelven elérhető