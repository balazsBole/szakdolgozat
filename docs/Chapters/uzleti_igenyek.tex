Ebben a fejezetben bemutatom a \foreignlanguage{british}{Helpdesk} alkalmazás felé megfogalmazott üzleti igényeket.



\section{Funkcionális igények}
Az alkalmazásnak az alábbi funkcionális igényeknek kell megfelelnie.


\subsection{E-mail fogadása és küldése}\label{sec:email_fogadas_kuldes}
Az ügyfelektől érkező e-maileket az alkalmazás képes fogadni, hosszú távra megőrizni. Az ügyfelek számára formázott válasz e-mail küldhető.

A rendszernek képesnek kell lennie több e-mail cím kezelésére. A beérkező új üzeneteket --~a kapcsolódó üzenetszálon keresztül~-- a címzettnek megfelelő előre definiált sorhoz kell hozzárendelnie.



\subsection{E-mail szálak kezelése}
A rendszer által kezelt üzenetek szálakba rendezve érhetőek el. Egy szál az ügyfél és a felhasználó közötti üzenetváltásokból épül fel.

Az üzenetszálakra vonatkozó összes adat historikusan lekérdezhető, státuszuk \aref{fig:statusz_diagram} ábrán definiált útvonalaknak megfelelően változtatható.


\begin{figure}[hbt] 
	\centering
	\includegraphics[width=0.85\textwidth]{statusz_diagram_drawio.pdf}
	\caption{Az e-mailszálak státuszváltozásai}
	\label{fig:statusz_diagram}
	\floatfoot{Forrás: saját ábra}
\end{figure}



\begin{description}
	\item[\textsc{open}] vagy nyitott állapotba kerül az üzenetszál, ha új e-mail érkezik az ügyféltől, vagy ha a felhasználó átállítja az e-mail szál státuszát \textsc{open}-re. A felhasználók csak \textsc{open} állapotú e-mail szálra képesek válaszüzenet küldeni.
	
	\item[\textsc{resolved}] vagy lezárt állapotú egy e-mail szál, ha a felhasználó a válasz üzenet küldése során a szálat \textsc{resolved} státusszal jelöli meg. A felhasználó ezzel az állapottal jelezheti, ha szerinte az üzenetszál lezárható, egy e-mail szál lezárt státuszban marad mindaddig, míg az ügyféltől nem érkezik kapcsolódó üzenet, ekkor az üzenetszál visszakerül \textsc{open} státuszba.
	
	\item[\textsc{clarification}] vagy tisztázandó állapotba kerül az üzenetszál, ha a felhasználó a válasz üzenet küldése során a szálat \textsc{clarification} státusszal jelöli meg. A felhasználó ezzel az állapottal jelezheti, ha szerinte az üzenetszál még nem zárható le, a továbblépéshez több információ, vagy az ügyfél általi további tisztázás szükséges.
	
	A \textsc{clarification} státuszú üzenetszál állapota a felhasználó által, kézzel is állítható. Abban az esetben, ha az ügyféltől válaszüzenet érkezik, akkor a szál státusza automatikusan \textsc{open} állapotúra változik.
	
	\item[\textsc{change\textunderscore queue}] vagy sor váltására várakozó állapottal jelezheti a felhasználó, ha úgy érzi, hogy az üzenetszál másik sorba, másik felhasználóhoz tartozik. 
	
	A \textsc{change\textunderscore queue} állapotú üzenetszálakat az adminisztrációs jogkörrel rendelkező felhasználók megvizsgálják, a sor váltására vonatkozó igényt elbírálják, és a megfelelő sor, felhasználó és státusz kiválasztásával, az üzenetszál állapotát véglegesítik.
	
\end{description}




\subsection{Több felhasználó}\label{sec:tobb_felhasznalo}
A rendszert egyszerre több felhasználó használhatja. Regisztrációjukat az alkalmazásba, és saját adataik kezelését egyénileg végzik.

A felhasználói felületen elérhető funkciókat \aref{fig:funkcio_diagram} ábrán foglaltam össze.

\begin{figure}[hbt] 
	\centering
	\includegraphics[width=0.85\textwidth]{funkcio_diagram_drawio.pdf}
	\caption[Elérhető funkciók]{Elérhető funkciók jogosultság szerint csoportosítva}
	\label{fig:funkcio_diagram}
	\floatfoot{Forrás: saját ábra}
\end{figure}


Minden felhasználó csak a saját üzenetszálait kezelheti, csak azokra válaszolhat. Pontosan egy,  \aref{sec:email_fogadas_kuldes} pontban említett sorhoz tartozik, és csak az ugyanabba a sorba tartozó e-mail szál rendelhető hozzá. A számára kijelölt szálakat képes --~a saját során belül~-- más felhasználóhoz rendelni. 

A felhasználók eltérő jogkörökkel rendelkezhetnek. Az adminisztratív jogkörrel rendelkező felhasználó végzi az új e-mailszál felhasználóhoz rendelését, valamint a \foreignlanguage{british}{\textsc{change\textunderscore queue}} státuszban (\ref{fig:statusz_diagram} ábra) lévő üzenetszálak új sorba irányítását.


A konfigurációs jogkörrel rendelkező felhasználó feladata az alkalmazásban használt jogkörök (\foreignlanguage{british}{\emph{role}}-ok) kezelése, felhasználók adminisztrációs jogkörhöz rendelése. Lehetősége van továbbá az autentikációs események megtekintésére, jelszó visszaállítására és más felhasználók kizárására vagy megszemélyesítésére (\foreignlanguage{british}{\emph{impersonate}}).


\section{Nem funkcionális igények}
Az alkalmazásnak az alábbi nem funkcionális igényeknek kell megfelelnie.
	

\subsection[Skálázhatóság]{Horizontális skálázhatóság}
A kiszolgálandó kliensek száma napi és havi szinten is eltérő. Az év egyes időszakaiban nagyobb volumenű ügyfél-interakció prognosztizálható. A hibatűrés javítása, és a megnövekedett forgalom érdekében --~ezekben az előre meghatározott időszakokban~--  horizontális skálázódás szükséges.


\subsection{Granuláris felosztottság}\label{sec:granularitas}
A \foreignlanguage{british}{helpdesk} alkalmazást használó ügyfélszolgálat munkaórákban a legaktívabb, míg az e-maileket küldő ügyfelek hétvégente és hétköznap munkaórákon kívül a legaktívabbak.

A hosszútávú tervekben szerepel a \foreignlanguage{british}{helpdesk} alkalmazás és a belső céges levelezés integrálása.

A fenti két szempont miatt célszerű a megvalósítandó funkciók minél nagyobb mértékű szeparálására törekedni.


\subsection{Mérhető indikátorok}\label{sec:indikatorok}
A rendszernek átlagosan $100$ felhasználót kell kiszolgálnia másodpercenként. A várható csúcsteljesítmény $5~000$--$10~000$ lekérdezés másodpercenként. A felhasználók száméra elfogadható legnagyobb válaszidő 3 másodperc/lekérés.


\subsection[L10N]{L10N\footnote{A nyelvi lokalizációt, vagy kulturális beágyazást szokás L10N-ként rövidíteni.}}
A felhasználói, adminisztratív és karbantartói felületek angol nyelven érhetőek el. A magyar, vagy attól eltérő egyéb nyelv kezelése nem szükséges.

Tervezés során, a honosításhoz, vagy többnyelvűsítéshez szükséges szempontokat nem kell figyelembe venni.
