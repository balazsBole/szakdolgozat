\section{A deploymentről}

A helpdesk alkalmazás szervizei úgy lettek kialakítva, hogy képesek legyenek egymástól függetlenül, akár több példányban is működni. Ezáltal költséghatékonnyá téve a működést, leegyszerűsítve a hibatűrést és lehetővé téve a változó terhelés miatti skálázhatóságot.

Ám amíg a docker konténerek egy host gépen futnak, és egy erőforráson osztoznak, soha nem lehet gazdaságosan megoldani a skálázást, és nem tud a rendszer felkészülni a számítógép kiesésére.

A következő logikus lépés tehát az alkalmazás \textit{cluster}re migrálása. A docker natívan támogatja a Microsoft Azure és az Amazon \cite{docker_website_deploy_ECS} szolgáltatókat. Így tehát a kód és a beállítások módosítása nélkül lehetséges az alkalmazás \textit{cluster}esítése docker swarmmal.


\section{A kódról}
A helpdesk alkalmazásba --architektúrája miatt-- könnyű új funkciót fejleszteni. A most működő modulok mind lazán kapcsolódnak egymáshoz, így könnyű egy teljesen különböző, akár eltérő programnyelven íródott új funkció integrálása.

Mivel az összes technikai megkötés csupán a protokollok megvalósítása, nyugodtan lehet az új funkció tervezésénél a feladathoz választani a programnyelvet vagy a programozási módszertant is. 

Ugyanígy, a laza kapcsolatok, és jól definiált határok miatt, egyszerű egy-egy modult teljesen lecserélni, vagy más nyelven, más technológiával újraírni.

Mivel egy szerviz egy feladattal foglalkozik, ha például le kell cserélni a fontendet, akkor az új felhasználó felületen csak a megjelenítéssel kell foglalkozni, az üzleti funkciók megvalósítása a backend feladta, így azok továbbra is változatlanok maradnak.

Ugyanez nem csak a szervizek, hanem a kód szintjén is igaz. A hexagonális architektúra miatt, az adatbázis --mint külső függőség-- könnyen cserélhető. 




