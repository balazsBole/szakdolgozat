\section{Mikroszerviz infrastruktúra}
\subsection{nginx}
static html-t szolgál ki a frontendnek -- lightweit application server

routingot valósul meg, rajta keresztül lehet elérni az auth*-ot és a helpdesk backendet

cache-t valósít meg a backen és a frontend között

\subsection{docker konténerizáció}
round-robin dns, docker compose, scale, images

\subsection{metrikák}
prometheus scraper, graphana, eurekát is itt használom leginkább






\section{E-mail kliens}
kafka producer, imap és smtp kliens

\subsection{E-mail szabvány}
	rfc5322: 
	messageId
	replyTo
	refereneces







\section{helpdesk backend}
Class Diagram

\subsection{springBoot}
default behavior, DI
security, 	

\subsection{data persistance layer}
hibernate, liquibase, Hibernate envers (Auditlog)

\subsection{egyéb eszközök}
openApi dokumnetáció, mapstruct, hibernate, lombok






\section{helpdesk frontend}
MVC szerint van szeparálva a kód 	

\subsection{RxJs store}

\subsection{Komponensek}
material, quill

\subsection{deploymnet}
static html-lé fordul a kliens oldalán fut a frontend






\section{keycloak}	
\subsection{JWT-token}
\subsection{role-ok}
\subsection{admin felületről valami?}