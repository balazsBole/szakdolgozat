\section{Mikroszerviz infrastruktúra}

\subsection{Nginx}
Az nginxnek három elkülönülő szerepe van:

\begin{itemize}
	\item{A \foreignlanguage{british}{helpdesk frontend} alkalmazásszervereként működik (lásd \ref{sec:angular} pont)}
	
	\item{\foreignlanguage{british}{Routing}ot valósít meg, rajta kerestül érhető el a \foreignlanguage{british}{helpdesk backend} és a \foreignlanguage{british}{keycloak} szerviz}
	
	\item{\foreignlanguage{british}{HTTP cahce}-ként működik a frontend és a backend között, illetve a frontend és a keycloak között}
\end{itemize}

A loadbalancer funkcionalitás a \foreignlanguage{british}{docker round-robin DNS}-én (\ref{sec:docker}) keresztül valósul meg.


\subsection{Docker konténerizáció}\label{sec:docker}
Az alakalmazás összes szervize saját docker konténerben fut. A docker konfigurációs leírása a \textit{docker-compose.yml} állományban van. A \textit{docker-compose} parancs ez alapján indítja el az alkalmazást, hozza létre a saját alhálózatát, valósítja meg a hálózaton belüli DNS-funkciót.

A konténerek skálázása  is a dockeren keresztül (\textit{docker-compose --scale}) valósul meg.



\subsection{Metrikák}
A springes alkamazásaim egy-egy HTTP endpointon keresztül érhetőek el a prometheus számára (\textit{\mbox{/actuator/proemtheus}}) és induláskor beregisztrálják magukat az eureka\footnote{Az Eureka a Netflix által fejlesztett \textit{discovery server}. Feladata az összes kliens port és ip adatának nyilkvántartása.} szerverbe.

A prometheus\footnote{A Prometheus egy open source monitorozó eszköz. 15 másodpercenként lekérdezi a szervizek állapotát.} az eurekán keresztül találja meg az instanceokat, és gyűjti össze a metrikákat. Az alkalmazások információt küldenek a Kafka konnektorukról, REST interfészükről és az adatbázis kapcsolatukról\footnote{HikariCP-t használok JDBC kapcsolathoz}.

A Prometheus által összegyűjtött adatokat grafanában\footnote{A Grafana egy open source elemző és megjelenítő web alkalmazés} ábrázolom.






\section{E-mail kliens}
Az e-mail kliens szerepe az üzenetek küldése és fogadása egy meghatározott e-mail címről. Feladata a külső protokollok leválasztása az alkalmazástól. Irányítja és karbantartja az IMAP és SMTP szerverrel való kapcsolatot.

A két irányú kommunikáció megvalósulása:
\begin{itemize}
	\item az IMAP-on keresztül fogadott e-mailt az \textit{email.in.v1.pub} kafka topicba írja,
	\item a saját --e-mail cím specifikus-- topic-jából kiolvassa az üzenetet és továbbítja  az SMTP szerver felé.
\end{itemize}


\subsection{E-mail szabvány}
Az elküldött üzenetek megfelelnek az \textit{rfc5322} szabványnak, különös tekintettel a 3.6.4. pontban~\cite{rfc5322_Identification_Fields} meghatározott mezőkre:

\begin{description}
	\item[Message-ID] egy globálisan egyedi azonosító ami egyértelműen azonosítja az üzenetet,
	
	\item[In-Reply-To] válasz esetén értéke eredeti üzenet \textit{Message-ID}-ja,
	
	\item[References] azonosítja az üzenet szálat, értéke az eredeti üzenetek \textit{Message-ID}-jai vesszővel elválasztva.
\end{description}




\section{helpdesk backend}
Class Diagram


kiszolgálja a frontendet resten keresztül

adatbázissal van kapcsolatban

olvassa és írja a kafka topicot.


\subsection{springBoot}
default behavior, DI
security, 	

\subsection{data persistance layer}
hibernate, liquibase, Hibernate envers (Auditlog)

\subsection{egyéb eszközök}
openApi dokumnetáció, mapstruct, hibernate, lombok






\section{helpdesk frontend}
MVC szerint van szeparálva a kód 	

\subsection{RxJs store}

\subsection{Komponensek}
material, quill

\subsection{deploymnet}
static html-lé fordul a kliens oldalán fut a frontend






\section{keycloak}	
\subsection{JWT-token}
\subsection{role-ok}
\subsection{admin felületről valami?}