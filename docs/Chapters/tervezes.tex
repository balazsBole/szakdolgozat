Ebben a fejezetben összegyűjtöttem a tervezés során felmerült kérdéseket, az azokra adott válaszaimat, és az egy-egy eszköz kiválasztása során figyelembe vett alternatívákat.



\section{Webszerver}
A megfelelő webszerver kiválasztása során a három legelterjedtebb\cite{W3Techs_Usage_statistics_of_webservers} webszerveret --Apache, Nginx és az Internet Information Services-- vettem számításba. A főbb szempontjaim a következők voltak:
\begin{itemize}
	\item legyen open source, vagy ingyenesen használható, ezzel minimalizálva a költségeket,
	\item lehetőleg minél kevesebb erőforrást használjon fel,
	\item úgy hogy közben elegendő csak statikus tartalmat kiszolgálnia.
\end{itemize}

Az IIS\footnote{Internet Information Services} --a többi open source alternatívával ellentétben-- nem felel meg az első pontnak, mivel Windows NT licenc alá tartozik.

Az Nginx alapvetően jobb az Apache-nál a statikus oldalak, a párhuzamos lekérések kiszolgálásában~\cite{nginx_performance}, emiatt szívesen használják a két alkalmazást együtt. Ilyenkor az Nginx egy fordított proxyként kezeli az ügyfeleket és a statikus tartalmak kiszolgálását, a dinamikus tartalmakat pedig továbbítja az Apache szerver felé.

Mivel a webszervernek elegendő statikus tartalmat kiszolgálni --hiszen a helpdesk frontend egy egyszerű statikus HTML-oldallá fordul (lásd \ref{sec:angular}~pont)-- így adja magát, hogy webszerverként az Nginx-et használjam.




\section{Adatbázis}
A megfelelő relációs adatbázis kiválasztása során, a következő három alternatívát vettem számításba:
\begin{itemize}
	\item Oracle DBMS,
	\item MySQL,
	\item és PostgreSQL.
\end{itemize}

Az Oracle adatbázist a drága licenc miatt nem tartom jó választásnak, helyette inkább egy open source megoldást választanék.

Az alkalmazás működése szempontjából lényeges funkciók a MySQL és a PostgreSQL adatbázisban is elérhetőek.

A backend alkalmazásban az adatokat JPA-n keresztül kezelem, így --a megfelelő adatbázis kiválasztásában-- a legfontosabb szempontnak a két adatbázis Hibernate-en keresztül elért teljesítményét tartom.

\subsection{Teljesítményvizsgálat}
Az ObjectDB által készített teljesítményvizsgálat\cite{JPA_benchmark} a saját --memóriában futtatott-- adatbázisuk teljesítményéhez hasonlítja többi alkalmazás teljesítményét (\ref{eq:normalizalt} egyenlet). Így a normalizált mérőszámok egymással összevethetőek.

\begin{equation}
\textrm{vizsgált rendszer mérőszáma} = 100 \cdot
\frac{\textrm{vizsgált rendszer teljesítménye}}{\textrm{ObjectDB teljesítménye}}
\frac{[\textrm{művelet}/s]}{[\textrm{művelet}/s]}
\label{eq:normalizalt}
\end{equation}

\Aref{tabl:teljesitmenyvizsgalat} táblázatban --a dokumentált eredményekből\cite{JPA_benchmark}-- összegyűjtöttem a vizsgálat szempontjából mérvadó adatokat. Mindegyik mérés külön példányon, egyesével, egyenként $100~000$ véletlenszerű entitás létrehozásával készült. \Aref{eq:normalizalt} egyenletből következik, hogy egy mérőszám minél magasabb értékű, a rendszer annál jobb teljesítményűnek számít.

\begin{table}[hbt]
	
	\begin{tabular}{lc|c}
		Tesztesetek & MySQL adatbázis & PostgreSQL adatbázis \\\hline 
		
		Új adat létrehozás & 2,8 & 7,7\\ \hline
		Keresés elsődleges kulcs alapján  & 5,2 & 7,0\\ \hline
		Keresés szóeleji egyezés alapján & 2,9 & 20,7\\ \hline
		Meglévő adat módosítása & 1,2 & 5,4\\ \hline
		Meglévő törlése & 1,2 & 6,6 \\ \hline
		Tesztesetek átlaga & 2,7 & 9,1
	\end{tabular} 
	
	\caption{JPA teljesítményvizsgálata Hibernate implementációval és  MySQL illetve PostgreSQL adatbázissal }
	\label{tabl:teljesitmenyvizsgalat}
\end{table}

\Aref{tabl:teljesitmenyvizsgalat} táblázatban látszódik, hogy a PostgreSQL adatbázis a Hibernate implementációval átlagosan háromszor olyan jól teljesít, mint a MySQL adatbázis. Ám ha figyelembe vesszük, hogy a két leggyakrabban előforduló funkció az elsődleges kulcs valamint szóeleji egyezés alapján keresés, akkor egyértelmű hogy a PostgreSQL adatbázist érdemes választani.


\section{JPA implementáció}
A helpdesk backend által használt adatok kezelésére objektum-relációs leképzést használok, mert az ORM nagy mértékben leegyszerűsíti az adatbázis és az entitások egymásnak való megfeleltetését.


A számtalan --EclipseLink, Hibernate, Spring Data JPA, \dots-- ORM-et megvalósító JPA implementációk közül a Hibernate mellett döntöttem, mert
\begin{itemize}
	\item PostgreSQL adatbázissal hatékonyan működik együtt,
	\item és a JPA implementáción felül, az adatok auditálására szeretném használni a projekten.
\end{itemize}

Így a Hibernate használatával jelentős mennyiségű forráskód elkészítését és karbantartását lehet kiváltani, hiszen az adatok tárolásáráért, módosításáért, verziókezeléséért mind a Hibernate a felelős.

A használatával kiváltott munka mennyisége bőven meghaladja a megismerésével és beállításával eltöltött időt.


\section{Adatbázis migrációs eszköz}
Az eszköz célja az adatbázis verziókövetésének javítása. A megvalósítás alapja, hogy ha az adatbázis változásait egy szöveges dokumentumban tároljuk az alkalmazás forráskódjával együtt, akkor a verziókövető rendszerrel nyomon lehet követni a kód aktuális állapotához tartozó adatbázis-struktúrát is. Összesen két eszközt találtam és vizsgáltam meg: a Liquibase-t és a Flywayt.

\subsection{Liquibase}
A Liquibase egy --adatbázis sémaváltozások nyilvántartására és alkalmazására létrehozott-- nyílt forráskódú adatbázis-független könyvtár.
	
A sémaváltozásokat tudja kezelni SQL, XML, JSON és YAML formátumban is. A nem SQL formátumban tárolt változásokból adatbázis-specifikus SQL-t tud generálni, és szükség esetén akár automatikusan visszagörgetni is képes azokat.

A \textit{changelog}ban a változások könnyedén csoportosíthatóak, sorrendjük módosítható végrehajtásuk feltételhez köthető.


\subsection{Flyway}
A Flyway egy nyílt forráskódú, SQL-alapú, egyszerű, adatbázis nyilvántartó könyvtár.

A változások csak SQL formátumban hozhatóak létre, és sorrendjük az őket tartalmazó állomány sorszámától függ. A migráció során figyelembe vett állományok neveinek meg kell felelniük a Flyway által előírt szigorú szabályoknak.


\subsection{Felhasznált eszköz}
A Liquibase --a Flyway funkcióinak megvalósítása mellett-- sokkal több, szélesebb körű felhasználást tesz lehetővé. A visszagörgethetőség és a változások egyszerűbb csoportosíthatósága miatt célszerű a Liquibase-t választani.




\section{Monitorozás}\label{sec:metrikak_tervezes}
Hogy megtaláljam a legmegfelelőbb monitorozó eszközt, összevetettem a három leginkább használt alternatívát, a Prometheust, Grafanát és a Graphite-ot.

\subsection{Prometheus}
Prometheus egy open source monitorozó eszköz. Kifejezetten alkalmas a metrikák idősoros tárolására, gyűjtésére és megjelenítésére.

A Prometheus által létrehozott és használt PromQL\footnote{Prometheus Query Language} lekérdező nyelv segítségével könnyen lehet az adatokból táblázatot vagy grafikont készíteni. A PromQL-ben létrehozott kifejezések alkalmasak riasztások létrehozására is, ilyenkor a hibásnak tekintett eseményről --a Prometheus \textit{Alertmanager}-én keresztül-- képes a riasztást kiváltó helyzet elhárításában érintetteket értesíteni.

A megjelenítésre használt \textit{Console template} meglehetősen sok és szerteágazó funkcióval rendelkezik. A rendelkezésre álló átfogó dokumentáció ellenére is jelentős időt vesz igénybe a legalapvetőbb grafikonok elkészítése.

A Prometheus HTTP-n keresztül tudja a megfigyelt rendszerek állapotát szabályos időintervallumokban lekérdezni. Az integrációt nagyban segíti, hogy összeköthető \textit{service discovery} szerverekkel.

\subsection{Grafana}
Grafana egy open source adatelemző, megjelenítő és monitorozó eszköz.

Nem tárol vagy gyűjt adatot, de támogat többféle --Graphite, Prometheus, Influx DB, ElasticSearch, MySQL, PostgreSQL-- adatforrással való kapcsolatot. \textit{Pluginok}on keresztül továbbá támogatja a felhő alapú AWS Cloudwatch és OpenStack Gnocchi adatforrást is.


A Grafana legnagyobb előnye a \textit{dashboard}-jaiban rejlik. A felületen --mint egy műszerfalon-- egymás mellett rendszerezve helyezkednek el a különböző paneleken ábrázolt metrikák. Számtalan előre elkészített típusú panelből lehet válogatni, többek között elérhetőek hisztogramok, grafikonok, hőtérképek, táblázatok, riasztások, RSS- és naplóbejegyzés-olvasó felületek.

A Grafana oldalán ezen kívül számos előre elkészített és szabadon felhasználható \textit{dashboard} érhető el. A leggyakrabban használt alkalmazások metrikái újra használhatóan és szerkeszthetően hozzáférhetőek. Így a fejlesztést jelentősen leegyszerűsítve, elegendő csupán az alkalmazás specifikus paneleket letölteni és testre szabni.

\subsection{Graphite}
A Graphite idősoros tárolására és megjelenítésére alkalmas open source monitorozó eszköz. Összesen két feladatot lát el, a Whisper könyvtárral tárolja a neki küldött adatokat, valamint megjeleníti a tárolt adatot a webes felületén.

Az adatok tárolása nem olyan kifinomult mint a Prometheuson: passzívan képes csak fogadni az adatokat és nem rendelkezik a PromQL-hez hasonló lekérdező nyelvvel sem.

A beépített felülete --körülbelül a Prometheus-éhoz hasonlóan-- messze nem olyan részletes mint a Grafanáé. Egyszerű grafikonok és táblázatok ugyanúgy elkészíthetőek vele, de ezen kívül támogatja még \textit{dashboard}ok létrehozását.


\subsection{Megvalósítás}
A fenti ismeretek fényében egy hibrid megoldást választottam (lásd \aref{sec:metrikak} pontban). A Grafana könnyen integrálható a Prometheussal, így egyszerre tudom kihasználni a Grafana kifinomult megjelenítését és szerkeszthető \textit{dashboard}jait, valamint a Prometheus PromSQL-jét, kiforrott adatgyűjtési és tárolási módszereit.