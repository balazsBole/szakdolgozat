Ahogy az \foreignlanguage{british}{O'Really} által az év elején készített felmérésből \cite{OReally} is látszik, a mikroszerviz alapú alkalmazások egyre nagyobb népszerűségnek örvendenek. Egyre több cég szeretné lecserélni meglévő monolit rendszerét, vagy a szükséges új funkciókat a régebbi rendszertől függetlenül, hibrid rendszerben valósítana meg.

Mint az a \foreignlanguage{british}{wiredelta} cikkéből \cite{wiredelta} is látszik, a mikroszerviz architektúrának számtalan előnye van. Míg a nagyvállalati környezetben sokszor a folyamatos szállítási igény, vagy az egymástól függetlenül fejleszthető alrendszerek miatt döntenek emellett a technológia mellett, az én esetemben a legfontosabb szerepet a skálázhatóság, az újrafelhasználhatóság, és  az alacsony fenntartási költség játszotta.

Úgy gondolom, hogy nincs olyan technológia, ami minden problémára megoldást nyújtana. De úgy érzem hogy az ilyen elvek mentén kialakított alkalmazások, természetükből adódóan időtállóbbak lesznek. Ha el tudjuk érni, hogy egy alkalmazás valóban csak egy funkcióért kell hogy felelős legyen, azzal a problémamegoldás analitikus oldalát  emeljük rendszerszintre. 

Éppen ezért, a mikroszerviz architektúra legnagyobb előnye szerintem a rendszerezésből következik.