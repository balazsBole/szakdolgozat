A mikroszerviz alapú alkalmazások egyre nagyobb népszerűségnek örvendenek, derül ki az \foreignlanguage{british}{O'Really} által készített felmérésből~\cite{OReally}. Egyre több cég szeretné lecserélni meglévő monolit rendszerét, vagy a szükséges új funkciókat a régebbi rendszertől függetlenül, hibrid rendszerben valósítaná meg.

A \foreignlanguage{british}{wiredelta} a témában készített cikkében~\cite{wiredelta} összegyűjtötte a mikroszerviz alapú architektúra előnyeit. Míg a nagyvállalati környezetben sokszor a folyamatos szállítási igény, vagy az egymástól függetlenül fejleszthető alrendszerek miatt döntenek emellett a technológia mellett, az én esetemben a legfontosabb szerepet a skálázhatóság, az újrafelhasználhatóság, és az alacsony fenntartási költség játszotta.

Úgy gondolom, hogy nincs olyan technológia, ami minden problémára megoldást nyújtana. De úgy érzem hogy a mikroszerviz elvei mentén kialakított alkalmazások, természetükből adódóan időtállóbbak maradnak. Ha el tudjuk érni, hogy egy alkalmazás valóban csak egy funkcióért kell hogy felelős legyen, azzal a problémamegoldás analitikus oldalát  emeljük rendszerszintre. Az én meglátásom szerint pont ebben, a feladatok és felelősségek rendszerezésében rejlik a mikroszerviz architektúra valódi előnye.