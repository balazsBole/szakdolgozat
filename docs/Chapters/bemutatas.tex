A helpdesk alkalmazás \aref{ch:uzleti_igenyek}. fejezetben leírtaknak megfelelően szolgál ki három különböző e-mail címet:

\begin{itemize}
	\item a \textit{generic} sorhoz tarozó  \href{mailto:helpdesk.gdf@yandex.com}{\nolinkurl{helpdesk.gdf@yandex.com}}-ot, 
	\item a \textit{travel} sorhoz tarozó  \href{mailto:helpdesk.gdf.travel@yandex.com}{\nolinkurl{helpdesk.gdf.travel@yandex.com}}-ot,
	\item és a \textit{theater} sorhoz tarozó  \href{mailto:h.gdf.theater@gmx.com}{\nolinkurl{h.gdf.theater@gmx.com}}-ot.
\end{itemize}



\section{Alkalmazás elindítása}\label{sec:elinditas}
Az alakalmazás a \textit{start.sh} bash \textit{script}tel indítható el. A \textit{script} két dolgot csinál:
\begin{enumerate}
	\item a \textit{docker-compose} parancssal elindítja a docker \textit{container}eket~(\ref{sec:docker} pont),
	\item  ``helpdesk'' domain névvel hozzáadja az Nginx (\ref{sec:nginx} pont) IP címét a \mbox{\textit{/etc/hosts}} állományhoz.
\end{enumerate}

A \textit{script} indítása után a helpdesk alkalmazás elérhető a  \href{http://helpdesk}{http://helpdesk} domain alatt.


\section{Több példány}
A különböző szervizekből a terhelésnek megfelelően eltérő számú példány indul el:

\begin{itemize}
	\item a helpdesk-backendből három,
	\item a \textit{theater} sort kezelő email-kliensből egy,
	\item a \textit{travel} sort kezelő email-kliensből kettő,
	\item a \textit{generic} sort kezelő email-kliensből három,
	\item és a Kafka brókerből (\ref{sec:kafka_topics}) szintén három darab.
\end{itemize}

A példányok metrikáit (\ref{sec:metrikak} pont) nyomon lehet követni az erre a célra létrehozott Grafana oldalon (todo ábra). Az oldal elérhető a  \textit{Spring metrics} menüpont alatt.

Az todo ábrán csak a Grafana oldal legfelső néhány panele látható, az instance-okra lebontott legfontosabb mérőszámokkal:

\begin{itemize}
	\item a legfelső sorban a Java Virtual Machine, által akutálisan felhasznált Heap space,	
	\item alatta az aktuális REST lekérések száma,
	\item a harmadik sorban a feldolgozott Kafka üzenetek száma,
	\item míg az utolsó sorban a Trace log bejegyzések száma látható.
\end{itemize}


\section{Deployment}
A könyebb bemutathatóság érdekében a szemléletesebb szervizeket --hogy ne a docker daemon által kiosztott IP címen keresztül kelljen elérni-- a docker hálózaton kívül is elérhetővé tettem. 

A \textit{docker-compose} (\ref{sec:elinditas}) által elindított \textit{container}eket \aref{fig:deployment_diagram}. ábrán foglaltam össze. Az ábrán feltüntettem, hogy az adott \textit{container}t a \textit{localhost} melyik portján lehet elérni.


\begin{figure}[hbt] 
	\centering
	\includegraphics[width=0.85\textwidth]{deployment_diagram_drawio.pdf}
	\caption[Deployment diagram]{Deployment diagram}
	\label{fig:deployment_diagram}
	\floatfoot{Forrás: saját ábra}
\end{figure}


\section{E-mail fogadásának és küldésének folyamata}
\Aref{ch:felepites}. fejezetben \aref{fig:path_of_an_email}. ábrán bemutattam egy e-mail fogadásának elméleti útját. Most \aref{fig:email_send_receive_visible}. ábrán szeretném bemutatni hogyan követhető végig a rendszerben egy e-mail valódi útja.

E-mail fogadása:
\begin{enumerate}
	\item Egy teszt üzenet érkezik a \href{mailto:helpdesk.gdf.travel@yandex.com}{\nolinkurl{helpdesk.gdf.travel@yandex.com}} címre
	\textit{E-mail fogadásának és küldésének folyamata} tárggyal
	\item Az e-mail kliens kafka üzenetként publikálja az üzenetet az \textit{email.in.v1.pub} topicba (\ref{fig:email_client_send_kafka}. ábra).
	\item A backend megkapja a kafka üzenetet (\ref{fig:backend_receive_kafka}. ábra)
	\item A backend elmenti az új üzenet az adatbázisba (\ref{fig:datbase_received_email}. ábra)
	\item A felhasználói felületen (\ref{fig:frontend_read_email}. ábra) elérhető az új üzenet.
\end{enumerate}

\bigskip

E-mail küldése:
\begin{enumerate}
	\item A felhasználó elküldi a válaszát a felhasználói felületen (\ref{fig:frontend_send_answer}. ábra).
	\item A backend megkapja az üzenetet és eltárolja az adatbázisba (\ref{fig:database_answer}. ábra).
	\item A \textit{h.gdf.theater\textunderscore gmx.com.v1.pub} topicban megjelenik  (\ref{fig:kafka_topic_send_email}. ábra) a backend által publikált kafka üzenet
	\item A topicra feliratkozott e-mail kliens fogadja és továbbítja az üzenetet (\ref{fig:email_client_receives_kafka}. ábra).
\end{enumerate}
 

\begin{figure}
	\begin{subfigure}{.49\textwidth}
		\centering
		\includegraphics[width=.9\linewidth]{email_client_sending_kafka_message.png}  
		\caption{Az egyes instance kafka üzenet küld}
		\label{fig:email_client_send_kafka}
	\end{subfigure}
	\begin{subfigure}{.49\textwidth}
		\centering
		\includegraphics[width=.9\linewidth]{helpdesk_frontend_send_email.png}  
		\caption{A felületen válasz e-mailt küld a felhasználó}
		\label{fig:frontend_send_answer}
	\end{subfigure}
	
	\quad
	
	\begin{subfigure}{.49\textwidth}
		\centering
		\includegraphics[width=.9\linewidth]{backend_receiving_email_from_kafka.png}  
		\caption{Az egyes instance kafka üzenet fogad}
		\label{fig:backend_receive_kafka}
	\end{subfigure}
	\begin{subfigure}{.49\textwidth}
		\centering
		\includegraphics[width=.9\linewidth]{database_with_the_answer.png}  
		\caption{A válasz e-mail az adatbázisban}
		\label{fig:database_answer}
	\end{subfigure}

	\quad

\begin{subfigure}{.49\textwidth}
	\centering
	\includegraphics[width=.9\linewidth]{databse_incoming_email.png}  
	\caption{Az új e-mail az adatbázisban}
	\label{fig:datbase_received_email}
\end{subfigure}
\begin{subfigure}{.49\textwidth}
	\centering
	\includegraphics[width=.9\linewidth]{kafka_topic_with_the_sent_email.png}  
	\caption{A \textit{h.gdf.theater\textunderscore gmx.com.v1.pub} topic új üzenete}
	\label{fig:kafka_topic_send_email}
\end{subfigure}

	\quad

\begin{subfigure}{.45\textwidth}
	\centering
	\includegraphics[width=.9\linewidth]{helpdesk_frontend_receive_incoming_email.png}  
	\caption{A felületen elérhető az új e-mail}
	\label{fig:frontend_read_email}
\end{subfigure}
\begin{subfigure}{.45\textwidth}
	\centering
	\includegraphics[width=.9\linewidth]{email_client_sending_email.png}  
	\caption{Az egyes instance kafka üzenet fogad}
	\label{fig:email_client_receives_kafka}
\end{subfigure}

	\caption{E-mail fogadásának és küldésének folyamata során követhető lépések}
	\label{fig:email_send_receive_visible}
\end{figure}



\section{Adatbázis táblák}




\section{Apache Kafka}\label{sec:kafka_topics}
A kafka \textit{topic}ok és üzenetek elérhetőek és követhetőek a \textit{Kafka messages} menü pontja alatti Kafka Topics UI (\ref{fig:Kafka_Topics_UI}. ábra) eszközzel.

\begin{figure}[hbt] 
	\centering
	\includegraphics[width=0.9\textwidth]{Kafka_topic_ui.png}
	\caption{A Kafka Topics UI eszközzel követhetőek a kafka \textit{topic}ok üzenetei, partíciói és beállításai}
	\label{fig:Kafka_Topics_UI}
	\floatfoot{Forrás: saját ábra}
\end{figure}


A helpdesk alakalmazás összesen öt \textit{topic}-ot használ:
\begin{description}
	\item[email.in.v1.pub] Az összes beérkező  e-mailt tartalmazza.
	
	\item[helpdesk.gdf\textunderscore yandex.com.v1.pub] A  \href{mailto:helpdesk.gdf@yandex.com}{\nolinkurl{helpdesk.gdf@yandex.com}} címre küldött e-maileket tartalmazza.
	
	\item[helpdesk.gdf.travel\textunderscore yandex.com.v1.pub] A  \href{mailto:helpdesk.gdf.travel@yandex.com}{\nolinkurl{helpdesk.gdf.travel@yandex.com}} címre küldött e-maileket tartalmazza.
	
	
	\item[h.gdf.theater\textunderscore gmx.com.v1.pub] A \href{mailto:h.gdf.theater@gmx.com}{\nolinkurl{h.gdf.theater@gmx.com}} címre küldött e-maileket tartalmazza.
	
	\item[\textunderscore schemas] A Schemaregistry ebben a \textit{topic}ban tárolja az alkalmazásban használt Avro schemákat.
\end{description}


Az alkalmazás három kafka brókert futat egy clusterben. Továbbá minden üzleti funkcionalitást hordozó \textit{topic} --a \textunderscore schemas-on kívül mindegyik-- három particíóval és kettes replikációs faktorral lett létrehozva.
Így a kafka cluster egy bróker kiesése, vagy egy partíció sérülése esetén is működőképes marad.


\section{frontend}
egy-egy érdekesebb problémásabb felület
 képenyőképek? valami how to dokumentáció	
 gondolok itt az urlben állapottárolásra
 meg a user keresés leütöm vallback meg 


\section{eureka meg grafana}

